\documentclass{article}

\usepackage[utf8]{inputenc}
%\usepackage[bindingoffset=1.0cm,rmargin=2cm, lmargin=2cm, tmargin=2cm, bmargin=2.5cm, twoside]{geometry}%[bindingoffset=2cm, margin=1cm,textheight=21cm, twoside]
\usepackage{amssymb}
\usepackage{amsfonts}
\usepackage{graphicx}
\graphicspath{ {images/} }
\usepackage{makeidx}
\usepackage{comment}
\usepackage{amsmath}
\usepackage{setspace}
\usepackage{parskip}
\usepackage{dcolumn}
\usepackage[font=footnotesize,labelfont=bf]{caption}
\usepackage[hidelinks,bookmarksdepth=3]{hyperref}
\usepackage{bookmark} 
\usepackage[labelfont=bf,textfont=bf]{caption} % captions for figures
\usepackage[capposition=bottom]{floatrow} % notes for figures
\usepackage{subcaption}
\usepackage{appendix}
\usepackage{multirow}
\usepackage[
backend=biber,
style=apa,
sorting=nyt
]{biblatex}

%%%    

\hypersetup{
    colorlinks,
    linkcolor={black},
    citecolor={black}, 
    urlcolor={black}
}

\linespread{1.25}


% Margins
\textwidth=165mm %162
\textheight=235mm %210
\evensidemargin0pt
\oddsidemargin0pt
\topmargin-35pt

% Bibliography
\addbibresource{EC331.bib}


\begin{document}

\title{The aggregate effects of urban growth in England with endogenous productivity} 
\author{Joshua Muthu}
\date{\today}
\maketitle 

\begin{center}
\includegraphics[height=4.5cm]{images/logo_warwick.png}


    Department of Economics\\ 
    University of Warwick \bigskip
    
    Student ID: 2100282 \\
    Word count: 4,953  \bigskip
    
\end{center}

\begin{abstract}
This paper investigates the links between urban growth dynamics, city-level productivity, planning regulation, and aggregate economic growth. I extend Duranton and Puga's (2023) urban growth model by endogenising city-level productivity growth, incorporating variables such as human capital, research labour force, and diffusion from the most productive city. Using data from English urban areas, I test the model's key predictions: that planning regulations are stricter in larger cities, and that planning regulations are stricter in cities more constrained by natural geography. I find support for the former but not the latter. A counterfactual analysis suggests that relaxing planning regulations in England's ten largest cities between 2001-2021 would have increased per-capita income by 1.867\% and consumption by 0.375\%. The model provides several theoretical insights, however it problematically predicts scale effects in city growth, contrary to empirical evidence.
\end{abstract}

\thispagestyle{empty}
\clearpage
\pagestyle{plain}

\section*{Acknowledgments}

I extend my sincere thanks to my supervisor, Professor Arthur Galichère, for his guidance and support throughout this year. His seminars were useful (and fun), and he provided me with frequent, in-depth 1-on-1 support. I could not have done this without him! 

Thanks to Sam Bowman and Ben Southwood; it was my conversations with them that first encouraged me to pursue this topic. I would also like to thank Duncan McClements for his invaluable support, input, and advice.

Finally, a thank you to my friends and family for keeping me sane and always being there for me!

\vspace{0.75cm}

\subsection*{AI use declaration}
I used Claude 3.7 Sonnet to assist with producing and refining my R code, particularly the data cleaning, API calls, and plots. I also used it to assist with formatting this~\LaTeX~document, particularly the images and tables.

I used Claude 3.7 Sonnet, OpenAI o4-mini, and the Perplexity AI search engine to search for relevant supporting literature. 

\clearpage

\tableofcontents
\clearpage

\section{Introduction}
Researchers have argued that cities are a major driver of economic growth and development (Jacobs, 1969; Lucas, 1988; Marshall, 1997; Glaeser, 2011). Cities, via agglomeration benefits such as knowledge spillovers, drive productivity improvements and thus long-term economic growth (Leishman and Goel, 2024). 

However, in developed countries, city growth has slowed while house prices have skyrocketed, partially due to stricter planning (land-use) regulations (Glaeser, Gyourko and Saks, 2003; Glaeser, Gyourko and Saks, 2005). Such regulation is linked to reduced aggregate welfare and economic growth (Gyourko and Molloy, 2014; Herkenhoff, Ohanian and Prescott, 2018; Hsieh and Moretti, 2019). Aggregate (i.e. country-level) economic growth has also slowed in advanced economies (Cowen, 2011; OECD, 2018; Lacina and Shine, 2024). The UK, in particular, has experienced weak wage, productivity, and GDP growth since the financial crisis (Harari, 2024; Nabarro, 2024). Concurrently, the UK’s strict regulatory constraints on housing have caused house prices to skyrocket and have adversely impacted the aggregate economy (Cheshire and Sheppard, 2002; Barker, 2006; Schleicher, 2013, Hilber and Vermeulen, 2016). Indeed, politicians and business leaders in the US and UK have proposed the relaxation of restrictive planning regulations, arguing they drive high housing costs and act as a brake on growth (The White House, 2022; CBI, 2024; The Labour Party, 2024).

Motivated by the above, my research investigates the links between urban growth dynamics, planning regulation, and aggregate economic growth. In particular, I consider the role played by city-level productivity (a term used interchangeably with ``technology’’, ``innovation’’, and ``idea generation’’) growth. I therefore seek to answer the following research question: How do urban growth dynamics, planning regulation, and endogenous city-level productivity growth relate to each other, and what are their implications for aggregate economic growth in England?

I answer this question via a theoretical advancement: I incorporate endogenous productivity growth into an urban growth model, and discuss the theoretical insights this yields. I also estimate this model on English data and test some of its predictions: namely, that planning regulations are stricter in more populated (and productive) cities, and that planning regulations are stricter in cities that are more constrained by natural geography. Finally, I use the model to examine the effects of counterfactually relaxing planning regulation in the ten largest English cities between 2001 and 2021. This counterfactual provides a quantitative answer to the second part of my research question, regarding the aggregate effects of urban growth in England.

The existing theoretical model I expand upon is Duranton and Puga (2023) – henceforth DP. DP’s model, discussed further below, is comprehensive and micro-founded; planning regulations are set to maximise city residents’ welfare, by balancing the ``fundamental trade-off’’ between urban agglomeration benefits and costs (Fujita and Thisse, 2013, p. 2). 

However, DP treat productivity growth as exogenous. Given the potential importance of city-based productivity and technological growth for long-run growth, I endogenise this process. Indeed, DP themselves recommend such modelling as an avenue for future research. 

Thus, my research addresses how urban cost-benefit tradeoffs, planning regulation, and endogenous city-based productivity growth relate to urban growth, and the implications of this for aggregate income and economic growth in England.

\subsection{Literature review} \label{litrev}
My research stems from the field of urban economics. I also draw upon semi-endogenous growth literature to incorporate endogenous productivity into my urban growth model.

Duranton and Puga (2014) and Duranton and Puga (2015) provide comprehensive overviews of the urban growth literature. Alonso (1964), Mills (1967), and Muth (1969) were the originators of modern urban economic theory. They created the monocentric city model that captured key features like transport, land use, and population, and how they varied with distance from the city centre (Alonso, 1964; Duranton and Puga, 2014; Duranton and Puga, 2015; Turner, 2024). Though seminal, the original model relies on strong assumptions, such as cities being on a featureless plane, having a single centre to which all homogeneous households commute, and having competitive land markets with no restrictions (Alonso, 1964; Turner, 2024). More recent models have thus included extensions to relax these assumptions (Duranton and Puga, 2015). Crucially, these models have: (1) examined the tradeoff between agglomeration economies and urban costs, (2) incorporated land use regulation, and (3) been empirically tested (Alonso, 1964).

Henderson (1974) and subsequent models captured the tradeoff between agglomeration economies and urban costs (namely higher house prices and commuting costs). Hsieh and Moretti (2019) made a landmark development by incorporating planning regulation into their model, which is used by city residents to optimise this tradeoff. The regulations create a permitting cost for potential new residents, creating productivity and wage wedges between cities. This yields labour misallocation, reducing aggregate productivity and output. US data empirically supported Hsieh and Moretti’s model’s predictions.

Likewise, DP (2023) model the urban cost-benefit tradeoff, incorporating endogenously set planning regulation and considering the aggregate implications of urban growth. However, their modelling is richer than HM’s, explicitly modelling commuting costs as a function of city size, and creating a more advanced urban framework with endogenous agglomeration benefits. DP calibrate their model and validate its predictions on US data. Their model also replicates stylised facts about cities in the US, such as the fact that as aggregate population grows, new cities appear, and the proportion of the population living in rural areas declines (Black and Henderson, 1999; Henderson and Wang, 2007; Sánchez-Vidal, González-Val and Viladecans-Marsal, 2014). However, as mentioned, DP model productivity growth as exogenous: cities experience independently and identically distributed productivity shocks in each period. As the process of idea generation and technological growth is a vital driver of aggregate growth, DP recommend that future research should explicitly model such growth. My research draws on the semi-endogenous growth literature to do exactly that.

Romer (1990) made a seminal contribution to the endogenous growth literature, identifying that ideas, being non-rival, yield increasing returns to scale in production. However, his model predicted growth scale effects, whereby larger populations increase per-capita output growth, contrary to empirical evidence (Jones, 1999). Jones (1995) addressed this through a semi-endogenous growth model, incorporating diminishing returns to R\&D – supported by Bloom et al. (2020) which found that ideas are getting harder to find. Bloom et al. (2020) and Jones (2005) disaggregate productivity growth into components including researcher efficiency, research intensity, and allocation of human capital to research – components which I include in my productivity growth equation. Benhabib and Spiegel (1994) incorporated cross-country productivity diffusion from the most productive countries – which I also adapt into my model. 

My application of (semi-)endogenous productivity growth within urban growth models is novel. Glaeser et al. (1992) model city-specific technology growth and find tentative empirical evidence of knowledge spillovers driving employment growth, but do not incorporate this into a more comprehensive urban growth model. Turner and Weil (2025) disaggregate cities’ productivity into 3 factors: (1) city-invariant, time-specific; (2) city-specific scale effect (agglomeration economy); and (3) city- and time-specific. They find larger cities have higher per-capita output and research productivity (more patents per capita). My work disaggregates productivity more comprehensively, aligning with the above components in Jones’ and Benhabib and Spiegel’s work. 

Though higher populations are not associated with higher per-capita output growth on a national level, these scale effects are more ambiguous at a city level (Jones, 1999). Eeckhout (2004) and González-Val, Lanaspa and Sanz-Gracia (2013) find no correlation between population growth rates and population levels in cities, which is consistent with no scale effects existing for per-capita output growth. However, He, Zhong and He (2024) find a U-shaped relationship between city size and per-capita income \textit{levels} in China, which implies a positive relationship between population and income \textit{growth}. 
\section{The Model}

Many of my model’s equations are unchanged from DP’s original model: some details are therefore left to the Appendix (Sections \ref{appendix humk} and \ref{appendix eqm}).

\subsection{Outline and initial setup}

$i$ indexes cities, and $t$ time. $N_t$, the economy’s total population, consists of cities’ populations $N_{it}$ plus the rural population $N_{rt}$.

Individuals live for two periods; in the first, they are children cohabiting with their parents. In the second, they choose between staying in their deceased parents’ residence or moving elsewhere. In each period, all adults die, and all children grow up and have one child each. 
Then, cities’ productivity $A_{it}$ grows endogenously (Section \ref{productivity growth}), based on human capital, the research labour force, and technology diffusion. Through an abstracted political process, adult residents set their city’s population to optimise their welfare, doing so by setting planning regulations that impose a regulatory permitting cost $p_{it}$ on potential newcomers. If adults decide to move to a new city, they incur that city’s regulatory cost $p_{it}$, and pay to lease a plot of land for 1 period. This payment goes to the local government, which redistributes the rent equally across everyone in the city population in the form of public benefits.
After adults choose where to reside for their remaining 1 period of life, they accumulate human capital, commute to their job, and consume housing and the numéraire (final) good. Rural workers obtain their income at their place of residence and only consume the numéraire.

\subsection{Human capital accumulation} \label{humkacc}
Human capital accumulation occurs at a constant rate, and all workers in a city $i$ have the same level of effective human capital:
\begin{align}
    h_{it} \equiv h_tN^\beta_{it} = (1-\delta)b(\delta)\bar{h}_tN^\beta_{it} = b(\delta)h_{t-1}N^\beta_{it}.
\end{align}

$(1-\delta)$ is the fraction of the individual’s adult life spent working. The remaining factors represent 3 stages of human capital accumulation: \textbf{compulsory education} provides individuals with the previous generation’s average post-further education human capital, $\bar{h}_t$. Then, $\delta$ is the fraction of adult life spent in \textbf{further education}, which augments human capital through a function $b(\delta)$. Third, workers acquire \textbf{initial work experience} in the city that they inhabit as an adult. This increases human capital as positive function of city population: $N^\beta_{it}, \ \beta>0$. This aligns with De la Roca and Puga’s (2017) finding that the value of early work experience in a city increases with the city’s population. 

\subsection{Endogenous productivity growth} \label{productivity growth}

I define growth in the city-specific production amenity (a term I will use interchangeably with productivity) to occur as follows:
\begin{align}
    \frac{A_{it}}{A_{it-1}} = 1+\xi_{it-1}h_{it-1}^\eta l_{it-1}N_{it-1}A_{it-1}^{-\phi}+\pi\left(\frac{A^*_{t-1}}{A_{it-1}}-1\right). \label{productivity growth eqn}
\end{align}

The middle term mirrors the idea production function in semi-endogenous growth models (Jones, 2003; Jones, 2022). A share $l_{it-1} \in [0,1]$ of the city's population $N_{it-1}$ engages in research to improve the city's production amenities. $\xi_{it-1}$ captures the productivity of researchers – in the context of empirical estimation, $\xi$ acts as a ``residual’’, capturing unspecified variation in productivity growth. $h_{it-1}$ captures human capital per worker in the city (see Section \ref{humkacc}), augmented by some elasticity $\eta>0$. 

$A_{it-1}^{-\phi}$, where $\phi>0$, captures the fact that productivity growth has diminishing returns to the stock of existing productivity (i.e. ideas or technology) – the phenomenon that ``ideas are getting harder to find’’ (Bloom et al., 2020). 

The final term captures the phenomenon of productivity diffusion (i.e. catch-up growth): $A^*_{t-1} \equiv \max_{j} A_{jt-1}$ is the highest production amenity of all the cities. Mirroring Benhabib and Spiegel (1994), where less-productive countries catch up to the most productive country, this highest productivity diffuses to all other cities, modified by some parameter $\pi>0$. 

\subsection{City output and agglomeration benefits} \label{sec: y and agglom}

As in DP’s model, final output is produced under CRS and perfect competition. An endogenous mass $m_{it}$ of intermediate inputs, with constant elasticity of substitution $\frac{1+\sigma}{\sigma},\ \sigma>0$, produce output:
\begin{align}
    Y_{it} = A_{it}\left\{ \int_{0}^{m_{it}}\left [ q_{it}(\omega) \right ]^{\frac{1}{1+\sigma}} \mathrm{d}\omega \right\}. \label{Y_it}
\end{align}

The quantity of intermediate input $\omega$ used in production, $q_{it}(\omega)$, has only one input: human capital. As producers are symmetric, they each employ the same human capital quantity:
\begin{align}
    q_{it}(\omega) = q_{it} = \frac{(1-l_{it})h_t(N_{it})^{1+\beta}}{m_{it}}. \label{q_it}
\end{align}

$h_t(N_{it})^\beta$ is an individual’s human capital, so $h_t(N_{it})^{1+\beta}$ is the whole city’s stock. $l_{it}$ is the fraction of the population devoted to research (increasing productivity). The remainder $(1-l_{it}) \in [0,1]$ produces intermediate inputs.

Substituting Equation \ref{q_it} into Equation \ref{Y_it}:
\begin{align}
    Y_{it} = A_{it} \left [ m_{it}(q_{it})^{\frac{1}{1+\sigma}} \right ]^{1+\sigma} = A_{it}(m_{it})^\sigma (1-l_{it})h_t(N_{it})^{1+\beta}.
\end{align}

By assumption, entrepreneurial ideas arise proportional to the city’s human capital stock after further education but before early work experience ($h_t(N_{it})$). Early job experience is not assumed to be valuable for idea generation. Ideas allow for creating new intermediate input producers, or for updating existing producers. In each period, non-updated producers become obsolete and exit. Hence, the total number of intermediate producers is:
\begin{align}
    m_{it} = \rho h_tN_{it}, \label{entrep ideas}
\end{align}

where $\rho > 0$ is the proportionality constant. Note $l_{it}$ is absent; I assume all adult city residents (and thus total human capital), not just researchers or intermediate input producers contribute towards generating entrepreneurial ideas. This is supported by studies emphasising that \textit{total} city population predicts innovation and growth (Bettencourt et al., 2007; Glaeser and Kerr, 2009).

Per-capita output is thus
\begin{align}
    y_{it} = \frac{Y_{it}}{N_{it}} = \rho^\sigma A_{it}(1-l_{it})(h_t)^{1+\sigma}(N_{it})^{\sigma + \beta}. \label{output per worker}
\end{align}

Thus, the model focuses on human capital spillovers and entrepreneurship as key drivers of agglomeration economies, aligning with existing research (Moretti, 2004a; Moretti, 2004b; Gennaioli et al., 2013; Glaeser, Pekkala Kerr and Kerr, 2015). Larger cities have more human capital (through both population and higher early work experience returns), yielding more entrepreneurial ideas and therefore more intermediate input producers (Glaeser and Maré, 2001; Baum-Snow, 2012; De la Roca and Puga, 2017). The latter increases final output due to constant elasticity of substitution; $\sigma$ and $\beta$ are thus parameters denoting agglomeration benefits.

\subsection{Urban costs: housing and commuting}
Urban costs are increasing in city population. This model assumes cities are linear and monocentric. Land extends along the real line, of which only a subset is occupied by residents at $t$. 
Cities face different geographical constraints (Saiz, 2010; Nagy, 2023). This is denoted by $z_i>1$: thus, 1 unit of ``raw’’ land yields only $1/z_i$ units of housing-suitable land.

All dwellings are identical by assumption, each occupying 1 unit of housing-suitable land (and hence $z_i$ units of raw land). Housing consumption is thus fixed at one unit, so utility maximisation is equivalent to maximising consumption of the final good. For simplicity, the cost of housing is assumed to equal the cost of renting land (plus the cost of planning regulation, discussed below).

City residents commute to the city centre to work. The cost of commuting from a residence at a distance $x$ from the centre is:
\begin{align}
    T_{it}(x) = \tau_{it}x^\gamma = \tau_t(N_{it})^\theta x^\gamma. \label{commute costs}
\end{align}

$\gamma>0$ represents the elasticity of the commute cost with respect to $x$. $\tau_{it}$ is the commute cost per unit distance. $\theta \in (0,1)$ captures congestion; travel is more costly in more populous cities. $\tau_t$ captures changes in ``commuting technology’’ (i.e. residual commuting costs) over time, such as changes in commuters’ valuation of time in their vehicles, and their speed of travel. This modelling aligns with evidence that larger cities have higher average house prices and higher commute costs (Couture, Duranton and Turner, 2018; Combes, Duranton and Gobillon, 2019). House prices are also higher in city centres, where commute costs are lower (Alonso, 1964; Muth, 1969).

\subsection{Rural sector}
In rural areas $r$, output per individual is
\begin{align}
    y_{rt} = A_{rt}(N_{rt})^{-\lambda}.
\end{align}

$\lambda \in (0,1)$ captures decreasing returns to rural labour. This can be intuited as rural areas having a factor in fixed supply, such as arable land.

\subsection{Equilibrium} \label{num and size of cities}

The economy must be in spatial equilibrium: residents must weakly prefer staying in their current location to moving (both across and within cities). A new resident moving to $i$ (at a distance $x$ from the centre) from a rural area incurs a permitting cost $p_{it}$. They earn $y_{it}$ from working at the city centre, incurring a commuting cost $T_{it}(x)$. By assumption, land rents are paid to the city government, which distributes them equally to city residents as public benefits. Thus, the new residents receive $R_{it}/N_{it}$ (where $R_{it}$ is the city’s total land rent).

$P_{it}(x) \equiv z_iR_{it}(x)$ represents the new resident’s bid for a dwelling ($R_{it}(x)$ represents the bid per unit of raw land, of which $z_i$ units are needed to construct 1 housing unit). The new resident is willing to make a maximum bid such that they are indifferent between inhabiting the city and a rural area (where they obtain utility equal to rural consumption, $c_t=y_{rt}$):
\begin{align}
    y_{it} - T_{it}(x) - P_{it}(x) + \frac{R_{it}}{N_{it}} - p_{it} = c_t = y_{rt},\quad \forall x. \label{new resident utility}
\end{align}

Incumbent residents aim to maximise their utility by setting planning regulations that determine the city’s population. Their utility is equivalent to their final consumption, 
\begin{align}
    c_{it}=y_{it}-T_{it}(x)-P_{it}(x)+R_{it}/N_{it}. \label{consumption}
\end{align}
    
I show in Appendix \ref{appendix eqm} that incumbents’ maximisation problem can be written:
\begin{align}
    \max_{N_{it}} c_{it} = \rho^\sigma A_{it}(1-l_{it})(h_t)^{1+\sigma}(N_{it})^{\sigma+\beta} - \frac{1}{\gamma+1}\tau_t(z_i)^\gamma(N_{it})^{\gamma+\theta}. \label{max c}
\end{align}

The above maximisation problem is identical to the one in DP’s original model, except for the presence of $(1-l_{it})$ (intermediate input-producing labour share). There is a clear tradeoff between the agglomeration benefits of population, $(N_{it})^{\sigma+\beta}$, and its costs, $(N_{it})^{\gamma+\theta}$.

The first-order condition yields equilibrium city sizes
\begin{align}
N_{it}=\left(\frac{\rho^\sigma(\sigma+\beta)(\gamma+1)}{\gamma+\theta}\frac{A_{it}(1-l_{it})(h_t)^{1+\sigma}}{\tau_t(z_i)^\gamma}\right)^{\frac{1}{\gamma+\theta-\sigma-\beta}}. \label{eqmcitypop}
\end{align}

Thus, both consumption and equilibrium population (which is optimal from the perspective of incumbent residents) are increasing in $A_{it}$, and decreasing in $l_{it}$ and $z_i$. Thus, were it not for the permitting cost $p_{it}$, people would like to move to higher-consumption, more populous cities with higher productivity or a lower share of researchers. 

Though the $p_{it}$ ``wedge’’ allows city incumbents to enjoy higher consumption, it also equalises the consumption (welfare) of the \textit{marginal} residents in each city $i$ (i.e., those who were indifferent between moving to $i$ and staying in their previous location). Equivalently, the economy’s minimum consumption level, $c_t$ (consumption in rural areas and in the marginally populated city), plus a given city $i$’s permitting cost equals incumbent residents’ consumption in $i$: $p_{it}+c_t=c_{it}$. 

Rearranging equation \ref{eqmcitypop}:
\begin{align}
    \rho^\sigma A_{it}(h_t)^{1+\sigma} = \frac{\gamma+\theta}{(\sigma+\beta)(\gamma+1)}\frac{\tau_t(z_i)^\gamma (N_{it})^{\gamma+\theta-\sigma-\beta}}{1-l_{it}}. \label{rho A h}
\end{align}

Substituting Equation \ref{rho A h} into \ref{max c} yields $c_{it}$ as a function of population:
\begin{align}
    c_{it} = \frac{\gamma+\theta-\sigma-\beta}{(\sigma+\beta)(\gamma+1)}\tau_t(z_i)^\gamma (N_{it})^{\gamma+\theta}. \label{c ito N z params}
\end{align}

This can be substituted into $p_{it} + c_t = c_{it}$, yielding an equilibrium equation for $p_{it}$:
\begin{align}
    p_{it} = \frac{\gamma+\theta-\sigma-\beta}{(\sigma+\beta)(\gamma+1)}\tau_t(z_i)^\gamma (N_{it})^{\gamma+\theta} - c_t. \label{eqm perm}
\end{align}

The above yields two key predictions, in line with DP’s original model. First, planning costs are higher (i.e. regulations are stricter) in cities with a higher population (and therefore higher productivity and consumption). Second, planning regulations are stricter in more geographically-constrained cities, because planning and geographical constraints are complements. A given population increase imposes greater crowding (urban costs) in more geographically-constrained cities, while yielding the same agglomeration benefits. These predictions are tested below. Ancillary predictions are discussed in Appendix \ref{appendix eqm}.

The equilibrium equations for $c_{it}$ and $p_{it}$ (Equations \ref{c ito N z params} and \ref{eqm perm}) in my modified version of DP’s model are identical to those in DP’s original model. The equilibrium $N_{it}$ equation (\ref{eqmcitypop}) is also almost identical, save for the $(1-l_{it})$ factor that does not feature in the original. Thus, endogenising productivity growth does not dramatically alter the model’s equilibrium values in a given time period, because only the level of $A_{it}$, not its growth rate, features in the equations.

\subsection{Growth trajectories} \label{urb growth}
Though endogenising $A_{it}$ growth does not affect key variables’ equilibrium values in each period, it does affect their growth rate. Obtaining log population change from Equation \ref{eqmcitypop}, with $\Delta$ indicating a 1-period difference:
\begin{align}
    \Delta\ln N_{it} = \frac{1}{\gamma+\theta-\sigma-\beta}\left[\Delta\ln A_{it} + \Delta\ln(1-l_{it}) + (1+\sigma)\Delta\ln h_t - \Delta\ln \tau_t \right].
    \label{log pop change}
\end{align}

City-specific productivity $A_{it}$ (the production amenity) grows according to Equation \ref{productivity growth eqn}, which can be rearranged to give log productivity change:
\begin{align}
    \ln\left(\frac{A_{it}}{A_{it-1}}\right) = \Delta\ln(A_{it}) = \ln \left[ 1 + \xi_{it-1}h^\eta_{it-1}l_{it-1}N_{it-1}A^{-\phi}_{it-1}+ \pi\left( \frac{A^*_{t-1}}{A_{it-1}}-1\right)\right] \label{log productivity change}
\end{align}

As mentioned in Section \ref{humkacc}, human capital growth is constant: $\Delta\ln h_t = \Delta\ln h$. I also use DP’s assumption that ``commuting technology’’, $\tau_t$, grows at a constant rate $\Delta\ln \tau_t = \Delta\ln \tau$. Thus, \ref{log pop change} can be rewritten as follows:
\begin{align}
    \Delta\ln N_{it} = \frac{1}{\gamma+\theta-\sigma-\beta} \left\{ \ln \left[ 1 + \xi_{it-1}h^\eta_{it-1}l_{it-1}N_{it-1}A^{-\phi}_{it-1}+ \pi\left( \frac{A^*_{t-1}}{A_{it-1}}-1\right)\right]\right. \notag \\
    + \Delta\ln(1-l_{it}) + (1+\sigma)\Delta\ln h - \Delta\ln \tau \biggr\}.
    \label{log pop change ito growth rates}
\end{align}

Similarly, we can obtain log output per capita changes from Equation \ref{output per worker}:
\begin{align}
     \Delta\ln y_{it} = \Delta\ln A_{it} + \Delta\ln(1-l_{it}) + (1+\sigma)\Delta\ln h_t + (\sigma+\beta)\Delta\ln N_{it}. \label{log y change} 
\end{align}

Equivalently:
\begin{align}    
\Delta\ln y_{it} = \ln \left[ 1 + \xi_{it-1}h^\eta_{it-1}l_{it-1}N_{it-1}A^{-\phi}_{it-1}+ \pi\left( \frac{A^*_{t-1}}{A_{it-1}}-1\right)\right] \notag \\
    + \Delta\ln(1-l_{it}) + (1+\sigma)\Delta\ln h + (\sigma+\beta)\Delta\ln N_{it}. \label{log y change ito growth rates} 
\end{align}

DP’s original model assumed production amenities grew based on exogenous, \textit{iid} shocks: $$A_{it}=g_{it}A_{it-1},\  g_{it}\in(1,\infty),$$ thus yielding systematic ($\Delta\ln(h)$, $\Delta\ln(\tau)$) and random components ($\ln(g_{it})$) to population and output growth. By endogenising productivity growth, I obtain additional theoretical insights. 

First, rather than DP’s simplistic assumption of independent $A_{it}$ shocks, productivity growth is now a function of the previous period’s productivity level. This has a knock-on impact on output and population growth rates.

Second, $A_{it}$ growth positively impacts $\Delta\ln N_{it}$, and thus affects $y_{it}$ both directly and indirectly (via population growth).

Third, in DP’s original model, human capital only affected population and output growth through its own (constant) growth. In my model, cities’ human capital level impacts $y_{it}$ and $N_{it}$ through its effect on $A_{it}$ growth.

Fourth, as in other semi-endogenous growth literature, there is a tradeoff between allocating labour to production versus research (Romer, 1990; Jones, 2003). A larger researcher share ($l_{it}$) increases $A_{it}$ (and thus population and per-capita output) growth, but an increase in the non-researcher share ($\Delta\ln(1-l_{it})$) also increases both $N_{it}$ and $y_{it}$ growth.

However, unlike DP’s original model, my model predicts growth scale effects: \textit{ceteris paribus}, a higher initial population level predicts a higher $y_{it}$ and $N_{it}$ growth rate. This is problematic, as empirical evidence rejects the existence of scale effects in city population growth, and the evidence on scale effects in city per-capita income growth is ambiguous at best (Section \ref{litrev}). Furthermore, scale effects predict explosive city-level growth – which is clearly problematic.

Thus, like Romer’s (1990) initial endogenous growth model, my model yields some novel insights while being flawed due to its prediction of scale effects. Future research should try to prevent scale effects arising from the endogenous growth specification.

Endogenising productivity growth means that the evolution of variables like population and productivity are interlinked and highly parameterised. This increases the model’s complexity and the difficulty of estimating the parameters and variables involved – the components of the productivity equation, $h_t$, and $\tau_t$. My estimations, discussed below, only consider the equilibrium equations. Thus, estimating the above growth equations is a vital next step to empirically verify my model’s growth predictions.

\section{Model estimation on English urban areas} \label{modelest}
Having outlined my model, which provides a theoretical response to my research question, I now use English data to estimate and evaluate it. I use LSOA (Lower Super Output Areas: low-level geographical divisions of under 10,000 inhabitants) population data from the 2001, 2011, and 2021 censuses (ONS, 2003; ONS, 2013b; ONS, 2022b). I also use ONS lookup tables that link different geographic subdivisions (including across time), and local authority (LAD)-level planning application statistics (ONS, 2017; ONS, 2023a; ONS, 2023b; ONS, 2024c; MHCLG, 2025).

I first map all population data to the 2021 LSOA divisions, imputing missing values by extrapolating England’s overall population growth rate (ONS, 2024b). I then aggregate LSOA data to the built-up area (BUA) level – the ONS subdivision for urban areas – as BUA population data does not exist for earlier census years (ONS, 2022a; Find that Postcode, 2025). I similarly use LSOA land use data to find the mean share of non-developed land classed as ``forestry, open land and water’’ in each BUA, which I then use to estimate $z_i$ – the measure of cities’ geographical constraints (DLUHC, 2022). 

Topographic data defines BUA boundaries more restrictively than in past censuses (ONS, 2023c). This disaggregates agglomerated metropolitan areas (for example, the 2021 Manchester BUA is much smaller than the 2011 Greater Manchester BUA), so BUAs are imperfect proxies for urban agglomerations (ONS, 2013a). BUAs in Greater London are defined by London borough boundaries, which I combine to form a single Greater London BUA, to better represent London's urban agglomeration (\textit{ibid.}). 

I calculate the total number of ``major’’ and ``minor’’ dwelling permits issued between 2001-2021 in each LAD, imputing missing values through linear regression where possible, or, failing that, the median number of permits issued in neighbouring LADs. I then estimate BUA-level permit numbers on the basis of LADs’ permit numbers (as there are no direct permitting figures for BUAs), assigning a share of each LAD’s permit total to each BUA based on the proportion of the LAD’s population in the BUA. I calculate a BUA-level ``permitting rate’’ – the estimated total number of dwelling permits issued between 2001 and 2021 per 2001 city resident. As my research focuses on cities, I only consider the 467 BUAs with a 2021 population above 20,000 – classed by the ONS as ``medium’’ or larger (ONS, 2023c). I calculate $N_{rt}$, the rural population, as the difference between England’s total population and the total urban population (those living in BUAs) (ONS, 2024a). 

Unfortunately, the ONS only has region-level data on the labour share employed in research, so I assume all BUAs within a given region have the same research employment share (Belt, Ri and Akinremi, 2021).

In my estimations, I use DP’s parameter values (estimated on US data), and conduct robustness checks by varying them. I do not re-estimate the values using English data, as re-estimation is a highly complex and data-intensive process outside the scope of this project. However, doing so should be an avenue for future research, as parameters will vary across countries.
\vspace{0.75cm}
\begin{center}
\begin{tabular}{|c|c|}
\hline
Parameter & DP's estimated value \\
\hline
$\gamma$ & 0.07 \\
\hline
$\theta$ & 0.04 \\
\hline
$\sigma$ & 0.04 \\
\hline
$\beta$ & 0.04 \\
\hline
$\lambda$ & 0.18 \\
\hline
\end{tabular}
\captionof{table}{DP's parameter values, estimated on US data}
\label{tab:parameters}
\end{center}
\vspace{0.75cm}

To estimate $y_{it}$ and $c_{it}$, I impute $\tau_t$ (capturing changes in commuting technology). I do this by calculating Equation \ref{rho A h} as a function of $\tau_t$, and thereby $y_{it}$ and $c_{it}$ (Equations \ref{output per worker} and \ref{c ito N z params}) as functions of $\tau_t$. $\tau_t$ is then imputed such that the weighted average growth of $y_{it}$ across all BUAs equals England’s real GDP per capita growth over the same period: 20.283\% over 2001-2021 (ONS, 2025). Rural consumption is defined as $c_t = \min\{c_{it}\}$. 

This produces a fully-specified equilibrium model with values for consumption and income. I now test the model’s two key predictions, mentioned in Section \ref{num and size of cities}. Due to a lack of sufficiently granular data on house prices and construction costs, I do not test the ancillary predictions discussed in Appendix \ref{appendix eqm}.

\begin{table}[!htbp] \centering 
  \caption{Empirically testing model predictions} 
  \label{tab:reg tab} 
\begin{tabular}{@{\extracolsep{5pt}}lD{.}{.}{-3} D{.}{.}{-3} } 
\\[-1.8ex]\hline 
\hline \\[-1.8ex] 
 & \multicolumn{2}{c}{\textit{Dependent variable:}} \\ 
\cline{2-3} 
\\[-1.8ex] & \multicolumn{2}{c}{Reciprocal of permitting rate (2001-2021)} \\ 
\\[-1.8ex] & \multicolumn{1}{c}{(1)} & \multicolumn{1}{c}{(2)}\\ 
\hline \\[-1.8ex] 
 Log population & 15.082^{***} &  \\ 
  & (3.422) &  \\ 
  & & \\ 
 Log \% of geographically constrained land &  & -13.727^{**} \\ 
  &  & (5.402) \\ 
  & & \\ 
 Constant & -77.269^{**} & 48.452^{***} \\ 
  & (36.913) & (14.620) \\ 
  & & \\ 
\hline \\[-1.8ex] 
Observations & \multicolumn{1}{c}{467} & \multicolumn{1}{c}{467} \\ 
R$^{2}$ & \multicolumn{1}{c}{0.040} & \multicolumn{1}{c}{0.014} \\ 
Adjusted R$^{2}$ & \multicolumn{1}{c}{0.038} & \multicolumn{1}{c}{0.012} \\ 
Residual Std. Error (df = 465) & \multicolumn{1}{c}{56.253} & \multicolumn{1}{c}{57.021} \\ 
F Statistic (df = 1; 465) & \multicolumn{1}{c}{19.420$^{***}$} & \multicolumn{1}{c}{6.457$^{**}$} \\ 
\hline 
\hline \\[-1.8ex] 
\textit{Note:}  & \multicolumn{2}{r}{$^{*}$p$<$0.1; $^{**}$p$<$0.05; $^{***}$p$<$0.01} \\ 
\end{tabular} 
\end{table} 

Column (1) of Table \ref{tab:reg tab} and Figure \ref{fig:perm rate vs city pop} show that English data supports the prediction of stricter planning regulation in more populated cities: there is a significant positive relationship between city population and the reciprocal of the permitting rate.

\vspace{0.75cm}
\begin{center}
    \includegraphics{images/perm_rate_vs_city_pop.png}
    \captionof{figure}{Planning regulations and city population}
    \label{fig:perm rate vs city pop}
\end{center}
\vspace{0.75cm}

However, the prediction of stricter planning regulation in more geographically-constrained cities is not supported: Column (2) of Table \ref{tab:reg tab} and Figure \ref{fig:geog constr vs perm rate} reveals a \textit{negative} correlation between geographic constraints and the reciprocal of the permitting rate.
\vspace{0.75cm}
\begin{center}
    \includegraphics{images/geog_constr_vs_perm_rate.png}
    \captionof{figure}{Planning regulations and geographical constraints}
    \label{fig:geog constr vs perm rate}
\end{center}
\vspace{0.75cm}

Future research should consider several avenues of research to align the model’s predictions with the data. First, though the model predicts the clear, univariate relationships that I test, more robust testing should be undertaken. In particular, the reciprocal of the permitting rate is an imperfect measure of the strictness of planning regulations. It does not capture, for example, height limits, land use restrictions, size limitations, floor-area ratio constraints, nor the complexity and timelines of an approval process (Duranton and Puga, 2015). Testing the model's predictions with an improved measure of planning regulation may yield different results. Second, as mentioned, parameters should be re-estimated with English data, rather than using DP’s US-estimated values as I do. Failing that, the model itself may have to be refined to better fit the data.

\subsection{Counterfactual evaluation}
Despite the model’s theoretical and empirical flaws, I now use the model to examine a counterfactual, in order to quantify the aggregate effects of urban growth in England. I calculate the city-level and aggregate effects of relaxing planning regulation (reducing $p_{it}$) in England’s ten largest BUAs: London, Birmingham, Leeds, Liverpool, Sheffield, Manchester, Bristol, Leicester, Coventry, and Bradford. Specifically, I increase the cities’ permitting rate to England’s 75th percentile level – 0.02107.

The motivation for this is that planning regulations, while maximising incumbent residents’ welfare, are inefficient at a national level because they prevent newcomers from entering and thereby enjoying higher welfare. Thus, the largest, most productive cities are inefficiently small, and with many small, less-productive cities existing to house more of the population.

Table \ref{tab:main counterfactual} displays the counterfactual results: all cities, except Bristol (whose permitting level remains unchanged at 0.0278, as it was already above the 75th percentile), grow in population, from Birmingham (growing by 345,000) to Bradford (18,000). Consequently, both agglomeration economies and urban (commuting) costs rise; the former raises output per person, between 0.26\% in London and 2.43\% in Manchester; the later reduces consumption. However, consumption losses are very small: between -0.0005\% in London and -0.0405\% in Manchester. This trades off against the benefits for newcomers that emigrate from rural areas and smaller cities, thanks to reduced permitting costs: they experience consumption gains of 1.218\% (slightly smaller for emigrants from other cities, who had higher consumption than rural residents). 
\vspace{0.75cm}
\begin{table}[htbp]
\centering
\renewcommand{\arraystretch}{1.5}
\begin{tabular}{|c|c|c|c|c|c|c|}
\hline
City & \begin{tabular}[c]{@{}c@{}}Change in\\permitting\\rate\\2001-2021\end{tabular} & \begin{tabular}[c]{@{}c@{}}Baseline\\population\\2021\end{tabular} & \begin{tabular}[c]{@{}c@{}}Counterf.\\population\\2021\end{tabular} & \begin{tabular}[c]{@{}c@{}}Change in\\output per\\capita\end{tabular} & \begin{tabular}[c]{@{}c@{}}Change in\\consumption\\per capita --\\incumbents\end{tabular} & \begin{tabular}[c]{@{}c@{}}Change in\\consumption\\per capita --\\newcomers\end{tabular} \\ \hline
Greater London & 22.839\% & 8,808,271 & 9,101,976 & 0.2627\% & -0.0005\% & 1.2178\% \\ \hline
Birmingham & 305.683\% & 1,123,562 & 1,468,397 & 2.1644\% & -0.0321\% & 1.2178\% \\ \hline
Leeds & 138.628\% & 537,067 & 607,428 & 0.9898\% & -0.0067\% & 1.2178\% \\ \hline
Sheffield & 111.163\% & 505,453 & 533,794 & 0.4374\% & -0.0013\% & 1.2178\% \\ \hline
Liverpool & 148.074\% & 506,863 & 545,467 & 0.5889\% & -0.0024\% & 1.2178\% \\ \hline
Manchester & 172.611\% & 469,173 & 633,526 & 2.4317\% & -0.0405\% & 1.2178\% \\ \hline
Bristol & 0.000\% & 428,579 & 428,579 & 0.0000\% & 0.0000\% & 1.2178\% \\ \hline
Leicester & 86.822\% & 405,806 & 467,497 & 1.1386\% & -0.0089\% & 1.2178\% \\ \hline
Coventry & 278.648\% & 347,816 & 438,483 & 1.8704\% & -0.0240\% & 1.2178\% \\ \hline
Bradford & 34.983\% & 333,809 & 351,617 & 0.4167\% & -0.0012\% & 1.2178\% \\ \hline
\textit{Rural areas} & & 17,072,782 & 16,755,434 & 1.2178\% & & \\ \hline
\end{tabular}
\caption{Relaxing planning regulations in the ten largest English built-up areas}
\label{tab:main counterfactual}
\end{table}
\vspace{0.75cm}

The 38 lowest-welfare cities are entirely vacated by their residents, who move to the larger cities, and the rural population drops by over 317,300. Across the whole economy, per-capita consumption rises by 0.375\%, and per-capita income by 1.867\%. In Appendix \ref{appendix robustness}, I vary the parameters $\sigma$, $\beta$, $\gamma$, and $\theta$ as robustness checks: the changes in per-capita income and consumption are very similar to the above. Loosening planning restrictions between 2001-2021 thus provides fairly modest gains in net welfare.

\section{Conclusion}
This research examined how urban growth dynamics, planning regulation, and endogenous city-level productivity growth relate to each other, and their implications for aggregate economic growth in England. I find that productivity growth in cities drives growth in population and per-capita income. City residents set planning regulation endogenously to optimise the tradeoff between agglomeration benefits and urban costs. Such regulation maximises welfare for incumbent residents but negatively impacts aggregate economic outcomes.

My theoretical contribution is incorporating endogenous productivity into a model of urban growth (and its aggregate effects), based on Duranton and Puga (2023), yielding important theoretical insights. If productivity growth is a function of previous productivity, then output and population growth are similarly path-dependent. Population growth (which affects per-capita income) is a positive function of productivity growth, which therefore has both a direct and indirect effect on per-capita income. The same argument applies to human capital, a component of productivity growth. There is also a population and income trade-off between allocating labour to production versus research, aligning with endogenous growth literature (Romer, 1990; Jones, 2003). However, highly problematically, my model predicts scale effects: higher population levels predict higher population and income growth. These predictions are not borne out in the data (Section \ref{litrev}), and predict explosive city-level growth. Future research must modify the model to remove this scale effect prediction.

Empirical analysis of my model validates the prediction of larger cities having stricter planning regulations, however the prediction of stricter planning regulations correlating with tighter geographical constraints does not hold. Future research should thus test the models’ predictions more rigorously (particularly improving the measure of planning restrictiveness beyond the reciprocal of the ``planning rate’’), including the ancillary predictions that I was unable to test (see Appendix \ref{appendix eqm}). Future research should also re-estimate the model parameters ($\sigma, \beta, \gamma, \theta, \lambda$) using English data, rather than relying on DP’s US estimations as I do. If such work still fails to validate the model’s predictions, the theoretical foundations should be modified to better fit the English data. 

Despite the model’s theoretical and empirical flaws, I perform a counterfactual analysis to quantify urban growth's aggregate effects. I find that loosening planning regulations in England’s ten largest cities between 2001-2021 would increase per-capita income by 1.867\%, and per-capita consumption by 0.375\%. This validates the theoretical prediction that planning regulation creates ``wedges'' between cities that yield an inefficient spatial allocation of economic activity, constraining aggregate economic growth. However, the quantitative effect of such regulation on the aggregate economy is fairly small.

As this project focuses on equilibrium outcomes (actual and counterfactual), I did not empirically estimate the growth paths of variables like income and population. Isolating and estimating the components of these variables’ growth (such as $\xi$, $\eta$, $\phi$, and $\pi$) would be valuable for better understanding the relationship between city-level productivity, urban growth, and their aggregate effects.

\newpage
\nocite{*}
\printbibliography[heading=bibintoc]

\newpage
\section{Appendix} \label{appendix}

\subsection{Derivation of human capital} \label{appendix humk}

As mentioned, there are 3 stages of human capital accumulation. First, as children, individuals receive compulsory education and achieve the average post-further education human capital level of the previous generation, $\bar{h}_t$.

Second, as adults, each individual $j$ chooses what share $\delta^j_t$ of the unit of time of their one-period adult life to spend in further education. This raises $j$’s human capital to $b(\delta^j_t)\bar{h}_t$, where $b(\delta^j_t)$ is a learning function which captures how further education increases human capital. DP assume $b'(\delta^j_t)>0$ and $b(0) = 1$. The average level of human capital of the previous generation is $\bar{h}_t \equiv \frac{\int b(\delta^j_{t-1})\bar{h}_{t-1}\mathrm{d}j}{\int \mathrm{d}j}$. 

Third, after completing further education, workers acquire initial work experience in the location where they have chosen to spend their adult life, which raises $j$’s human capital stock to $b(\delta^j_t)\bar{h}_t(N^j_{it})^\beta$. Note the increase in human capital is a positive function of the population size of the city. This is in line with De la Roca and Puga’s (2017) findings that the value of early work experience increases with the size of the city in which it is acquired. DP assume this knowledge is more personal, and is not passed on to the next generation. DP also assume this experience is instantaneous; that is, it does not occupy a share of individuals’ working lives.

Following these 3 stages of human capital evolution, individuals work for the remaining $(1-\delta^j_t)$ share of their adult lives. $j$ therefore provides the following amount of effective human capital to their employer:
\begin{align}
    h^j_t = (1-\delta^j_t)b(\delta^j_t)\bar{h}_t(N^j_{it})^\beta.
\end{align}

As stated in Section \ref{sec: y and agglom} Equation \ref{entrep ideas}, entrepreneurial ideas arise in proportion to total human capital in the city after further education but before early work experience, $h_t(N_{it})$. Assume that individuals receive the rewards from those ideas in proportion to their human capital. Let $\varrho_{it}$ denote the reward from each of the $m_{it}$ entrepreneurial ideas (i.e. the number of intermediate producers created) in each period in city $i$. $\varrho_{it}$ is a function of the difference between the revenue and cost of a producer of an intermediate good $\omega$. I define $s_{it}(\omega)$ as the price of $\omega$. Minimising final production costs subject to the output level given by Equation \ref{Y_it} yields the following conditional input demand:
\begin{align}
    q_{it}(\omega) = \frac{\left[s_{it}(\omega)\right]^{-\frac{1+\sigma}{\sigma}}}{\left\{ \int_{0}^{m_{it}}\left[s_{it}(\omega')\right]^{-\frac{1}{\sigma}}\textrm{d}\omega'\right\}^{1+\sigma}}\frac{Y_{it}}{A_{it}}.
\end{align}

The price elasticity of demand for each intermediate producer is therefore $-(1+\sigma)/\sigma$. Marginal revenue is $s_{it}/(1+\sigma)$ -- as intermediate producers are symmetric, the $\omega$ index is superfluous. The marginal cost, given the production function in Equation \ref{q_it}, is the price per unit of human capital employed, which I denote by $w_{it}$. Equating marginal revenue and marginal cost, the price of any intermediate input is therefore:
\begin{align}
    s_{it}=(1+\sigma)w_{it}. \label{eqn: s_it}
\end{align}

From Equations \ref{q_it} and \ref{eqn: s_it}, the returns to each entrepreneurial idea (i.e., each intermediate producer's profit) is therefore:
\begin{align}
    \varrho_{it}=(s_{it}-w_{it})q_{it}=\sigma w_{it}\frac{(1-l_{it})h_t(N_{it})^{1+\beta}}{m_{it}}. \label{profit}
\end{align}

As the returns to these ideas accrue to individuals proportional to their human capital, the income accruing to worker $j$ is:
\begin{align}
    y^j_t=m_{it}\varrho_{it}\frac{(1-\delta^j_t)b(\delta^j_t)\bar{h}_t}{h_t(N_{it})}+w_{it}(1-\delta^j_t)b(\delta^j_t)\bar{h}_t(N^j_{it})^\beta. \label{indiv income}
\end{align}

$j$ chooses the amount of time to spend in further education, $\delta^j_t$, in order to maximise their income, $y^j_t$. Substituting Equation \ref{profit} into \ref{indiv income} gives the following optimisation:
\begin{align}
    \max_{\left\{\delta^j_t\right\}}y^j_t=(1+\sigma(1-l_{it}))w_{it}(1-\delta^j_t)b(\delta^j_t)\bar{h}_t(N^j_{it})^\beta.
\end{align}

Taking the first-order condition yields:
\begin{align}
    \frac{b'(\delta^j_t)}{b(\delta^j_t)}=\frac{1}{1-\delta^j_t}.
\end{align}

To ensure a unique solution where $\delta^j_t \in (0,1)$, $b(\cdot)$ is restricted to be log-concave, and $b'(0)>1$. Then, for all $i$ and $t$, all workers spend the same share of their lives $\delta^j_t=\delta$ in education. This results in the equilibrium level of effective human capital is the same for all workers in a given city,
\begin{align}
    h_{it} = (1-\delta)b(\delta)\bar{h}_tN^\beta_{it} = b(\delta)h_{t-1}N^\beta_{it}.
\end{align}

\subsection{Further analysis of the equilibrium} \label{appendix eqm}

In spatial equilibrium, residents are indifferent across locations within a city. Thus, Equation \ref{new resident utility} can be equated at $x=0$ (where $T_{it}(0)=0$) and any other $x$:
\begin{align}
    T_{it}(x) + P_{it}(x) = P_{it}(0). \label{intra-city spatial eqm}
\end{align}

Note, that differentiating \ref{intra-city spatial eqm} demonstrates that increases in marginal housing costs and decreases in marginal commute costs must offset each other:

\begin{align}
\frac{\textrm{d}P_{it}(x)}{\textrm{d}x}=-\frac{\textrm{d}T_{it}(x)}{\textrm{d}x}
\end{align}

The city's edge, $\bar{x}_{it}$, is endogenous: the point beyond which city residents aren't willing to bid for land above the land’s rent from the best alternative use $\underline{R}=R_{it}(\bar{x}_{it})$. Like DP, I assume $\underline{R}=0$ for simplicity, as non-urban land values are dramatically lower and more homogeneous than urban values (MHCLG, 2020). Thus, comparing $x=\bar{x}$ and $x=0$ as in Equation \ref{intra-city spatial eqm}, with $T_{it}(x)$ defined as in \ref{commute costs}:
\begin{align}
    P_{it}(0) = \tau_{it}(\bar{x}_{it})^\gamma. \label{city centre price}
\end{align}

Thus, using Equations \ref{commute costs}, \ref{intra-city spatial eqm}, and \ref{city centre price}, a given bid value $R_{it}(x)$ can be rewritten
\begin{align}
    R_{it}(x) = \frac{\tau_{it}}{z_i}(N_{it})^\theta(\bar{x}_{it}^\gamma-x^\gamma). \label{bid-rent for land}
\end{align}

Note, because the city is linear, housing is homogeneous, and geographic constraints are uniform across the city, the city’s edge $\bar{x}_{it}=z_iN_{it}$. Substituting this into \ref{bid-rent for land} yields the city’s total land rents:
\begin{align}
    R_{it} = \int_{0}^{z_iN_{it}}R_{it}(x)\mathrm{d}x = \frac{\gamma}{\gamma+1}\tau_t(z_i)^\gamma (N_{it})^{\gamma+\theta+1}. \label{total rents}
\end{align}

Equations \ref{output per worker}, \ref{commute costs}, \ref{intra-city spatial eqm}, \ref{city centre price}, \ref{total rents}, and $\bar{x}_{it}=z_iN_{it}$ can then be substituted into the final consumption equation \ref{consumption} to yield the maximisation problem, Equation \ref{max c}. Note that the second-order condition of Equation \ref{max c} requires $\gamma + \theta - \sigma - \beta > 0$ and positive city sizes $\sigma + \beta > 0$.

The above yields the following ancillary predictions: that house prices at the city’s edge are higher in larger cities; and that stricter regulation should correlate with a greater wedge between house prices and construction costs, because planning regulations limit new construction.

\subsection{Robustness check: varying parameters in the counterfactual} \label{appendix robustness}

\vspace{0.75cm}
\begin{table}[htbp]
\centering
\renewcommand{\arraystretch}{1.5}
\begin{tabular}{|c|c|l|c|c|c|}
\hline
\multicolumn{3}{|c|}{} & \multicolumn{3}{c|}{$\sigma + \beta$} \\
\cline{4-6}
\multicolumn{3}{|c|}{} & 0.04 & 0.06 & 0.08 \\
\hline
\multirow{6}{*}{$\gamma + \theta$} & \multirow{2}{*}{0.09} & Consumption & 0.3930975 & 0.3928097 & 0.3925211 \\
\cline{3-6}
& & Output & 1.266596 & 1.3964 & 1.545087 \\
\cline{2-6}
& \multirow{2}{*}{0.11} & Consumption & 0.3754309 & 0.3750666 & \textbf{0.3747013} \\
\cline{3-6}
& & Output & 1.505247 & 1.676711 & \textbf{1.866505} \\
\cline{2-6}
& \multirow{2}{*}{0.13} & Consumption & 0.3581074 & 0.3576625 & 0.3572161 \\
\cline{3-6}
& & Output & 1.695481 & 1.880585 & 2.081046 \\
\hline
\end{tabular}

\smallskip
\begin{flushleft}
\textit{Note:} The bold results, where $\gamma+\theta=0.11$ and $\sigma+\beta=0.08$, correspond to the baseline case in Table \ref{tab:main counterfactual}.
\end{flushleft}

\caption{Percentage changes in consumption and output per capita in England under different parameter values}
\label{tab:robustness check}
\end{table}
\vspace{0.75cm}

\subsection{Data sources}
All the data sources that I use are cited inline in Section \ref{modelest}, and indicated in the References section. All data comes from UK government bodies: the Office for National Statistics (ONS), the Ministry of Housing, Communities \& Local Government (MHCLG), and the former Department for Levelling Up, Housing and Communities (DLUHC). I also use \href{https://findthatpostcode.uk/#api}{Find that Postcode}, an API service that uses ONS data to provide information on ONS-defined geographic areas. In some cases, I conducted some minor initial cleaning of the raw CSV/Excel data files before loading them into R. For further details, please see the data files and R code provided. 

\end{document}
