\title{The aggregate effects of urban growth in England with endogenous productivity} 
\author{Joshua Muthu}
\date{\today}
\maketitle 

\begin{center}
\includegraphics[height=4.5cm]{images/logo_warwick.png}


    Department of Economics\\ 
    University of Warwick \bigskip
    
    Student ID: 2100282 \\
    Word count: 4,953  \bigskip
    
\end{center}

\begin{abstract}
This paper investigates the links between urban growth dynamics, city-level productivity, planning regulation, and aggregate economic growth. I extend Duranton and Puga's (2023) urban growth model by endogenising city-level productivity growth, incorporating variables such as human capital, research labour force, and diffusion from the most productive city. Using data from English urban areas, I test the model's key predictions: that planning regulations are stricter in larger cities, and that planning regulations are stricter in cities more constrained by natural geography. I find support for the former but not the latter. A counterfactual analysis suggests that relaxing planning regulations in England's ten largest cities between 2001-2021 would have increased per-capita income by 1.867\% and consumption by 0.375\%. The model provides several theoretical insights, however it problematically predicts scale effects in city growth, contrary to empirical evidence.
\end{abstract}

\thispagestyle{empty}
\clearpage
\pagestyle{plain}